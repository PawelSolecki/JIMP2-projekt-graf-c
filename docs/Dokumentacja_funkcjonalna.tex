\documentclass[]{article}
\usepackage{lmodern}
\usepackage{amssymb,amsmath}
\usepackage{ifxetex,ifluatex}
\usepackage{fixltx2e} % provides \textsubscript
\ifnum 0\ifxetex 1\fi\ifluatex 1\fi=0 % if pdftex
  \usepackage[T1]{fontenc}
  \usepackage[utf8]{inputenc}
\else % if luatex or xelatex
  \ifxetex
    \usepackage{mathspec}
  \else
    \usepackage{fontspec}
  \fi
  \defaultfontfeatures{Ligatures=TeX,Scale=MatchLowercase}
\fi
% use upquote if available, for straight quotes in verbatim environments
\IfFileExists{upquote.sty}{\usepackage{upquote}}{}
% use microtype if available
\IfFileExists{microtype.sty}{%
\usepackage[]{microtype}
\UseMicrotypeSet[protrusion]{basicmath} % disable protrusion for tt fonts
}{}
\PassOptionsToPackage{hyphens}{url} % url is loaded by hyperref
\usepackage[unicode=true]{hyperref}
\hypersetup{
            pdftitle={Specyfikacja funkcjonalna symulatora układu N-ciał: nsim},
            pdfauthor={Tsimafei Lukashevich, Paweł Solecki},
            pdfborder={0 0 0},
            breaklinks=true}
\urlstyle{same}  % don't use monospace font for urls
\IfFileExists{parskip.sty}{%
\usepackage{parskip}
}{% else
\setlength{\parindent}{0pt}
\setlength{\parskip}{6pt plus 2pt minus 1pt}
}
\setlength{\emergencystretch}{3em}  % prevent overfull lines
\providecommand{\tightlist}{%
  \setlength{\itemsep}{0pt}\setlength{\parskip}{0pt}}
\setcounter{secnumdepth}{0}
% Redefines (sub)paragraphs to behave more like sections
\ifx\paragraph\undefined\else
\let\oldparagraph\paragraph
\renewcommand{\paragraph}[1]{\oldparagraph{#1}\mbox{}}
\fi
\ifx\subparagraph\undefined\else
\let\oldsubparagraph\subparagraph
\renewcommand{\subparagraph}[1]{\oldsubparagraph{#1}\mbox{}}
\fi

% set default figure placement to htbp
\makeatletter
\def\fps@figure{htbp}
\makeatother


\title{Podzielenie grafu na części}
\author{Tsimafei Lukashevich, Paweł Solecki}
\date{01.03.2025}

\begin{document}
\maketitle


\section{Cel projektu}\label{header-n231}

Celem projektu jest stworzenie aplikacji dokonującej podziału grafu na określoną liczbę części przy minimalnej liczbie przeciętych krawędzi i zachowaniu równowagi w liczbie wierzchołków. Program działa w trybie wsadowym, umożliwiając konfigurację parametrów podziału. Dane wejściowe są przekazywane w formacie tekstowym, a wynik zapisywany w pliku tekstowym lub binarnym, co pozwala na ich ponowne wykorzystanie.

\section{Dane wejściowe}\label{header-n233}

Do poprawnego uruchomiena programu mamy zdefinować graf za pomocą pliku tekstowego, czyli mamy podać ścieżkę do pliku z grafem, będącym argumentem wymaganym.

Program jest uruchomiany w terminalu, gdzie wszystkie parametry powinny być przekazywane przez linię poleceń.

\section{Argumenty wywołania programu}\label{header-n256}

Program akceptuje następujące argumenty wywołania:

\begin{itemize}
\item
    \texttt{-p / -\/-parts} – liczba części podziału (np. -p 3, -\/-parts 3), domyślnie 2.
\item
    \texttt{-m / -\/-margin} – maksymalny margines procentowy (np. -m 15, -\/-margin 15), domyślnie 10\%.
\item
    \texttt{-o / -\/-output} – format wyjściowy (-o txt lub -\/-output txt dla pliku tekstowego, -o bin lub -\/output bin dla pliku binarnego), domyślnie tekstowy.
\item
    \texttt{-f / -\/-force} – pominięcie wbudowanego marginesu błędu, co pozwala na wykonanie podziału bez dodatkowych ograniczeń.
\item
    \texttt{-v / -\/-verbose} – szczegółowe logowanie przebiegu działania programu, w tym liczby iteracji i wyników pośrednich.
\item
    \texttt{-F / -\/-filename} – nazwa pliku wyjściowego (np. -\/-filename wynik.txt), domyślnie wynik jest zapisywany do pliku domyślnego w wybranym formacie.
\end{itemize}

Przykładowe wywołania programu:

\begin{itemize}
\item
  \texttt{./program\ graf.txt -\/-parts\ 3\ -\/-margin\ 20\ -\/-output\ txt}
  
  efektem będzie podzielenie grafu odczytanego z pliku \texttt{graf.txt} na \texttt{3} części tak, że liczba wierzchołków w powstałych częściach grafu nie będzie się różnić o więcej niż \texttt{20} procentowy oraz liczba przeciętych krawędzi będzie jak najmniejsza. Wynik będzie zapisany do pliku formatu \texttt{txt}.
\item
  \texttt{./program\ graf.txt -\/-parts\ 2\ -\/-force\ -\/-verbose\ -\/-filename\ wynik.txt}
  
  efektem będzie podzielenie grafu na \texttt{2} części bez ograniczeń marginesu błędu, z aktywnym szczegółowym logowaniem i zapisaniem wyniku do pliku \texttt{wynik.txt}.
\end{itemize}

\section{Teoria}\label{header-n279}
Podział grafu polega na podziale zbioru wierzchołków grafu na wzajemnie rozłączne grupy, przy jednoczesnym minimalizowaniu liczby krawędzi łączących te grupy. Celem podziału jest uproszczenie analizy danych lub zwiększenie efektywności obliczeń w systemach rozproszonych.

Wyróżnia się kilka podejść do podziału grafu:
\begin{itemize}
    \item 
    Podział losowy – przypisywanie wierzchołków do grup w sposób przypadkowy, co zapewnia równomierność, ale nie minimalizuje liczby przeciętych krawędzi.

    \item 
    Algorytmy zachłanne – iteracyjne przypisywanie wierzchołków na podstawie lokalnych informacji, takich jak liczba połączeń do już przypisanych wierzchołków.

    \item 
    Podział spektralny – oparty na analizie macierzy sąsiedztwa i jej wartości własnych, pozwalający na uzyskanie bardziej optymalnych podziałów kosztem większej złożoności obliczeniowej.

    \item 
    Metody heurystyczne – np. algorytm Kernighana-Lina, który stopniowo poprawia istniejący podział poprzez zamianę wierzchołków między grupami.
\end{itemize}

Podczas oceny jakości podziału istotne są takie kryteria jak równomierny rozkład wierzchołków między grupami, minimalizacja liczby krawędzi między grupami oraz modularność, czyli miara spójności wierzchołków wewnątrz grup.


\section{Komunikaty błędów}\label{header-n281}

Program nie przyjmuje agrumentow w trakcie działania więc mamy jedynie zadbać o poprawność danych wejściowych.

\begin{itemize}
\item
    Błędne definacja grafu: \texttt{Linia 29: Niepoprawna definicja krawędzi. Wczytano: "1".}
\item
    Błędna liczba podziałów: \texttt{Liczba podziałów musi być większa od 0 i mniejsza od liczby wierzchołków. Wczytano: "-1".}
\item
    Błędny margines procentowy: \texttt{Margines procentowy musi być w zakresie 0-100. Wczytano: "123".}
\item
    Nieznany format wyjścia: \texttt{Program obsługuje formaty txt i bin. Wczytano: "pdf".}
\item
  Masa ciała w danych wejściowych mniejsza/równa 0:
  \texttt{Linia\ 31:\ Masa\ ciała\ musi\ być\ większa\ od\ 0\ kg.\ Wczytano:\ "-2.1".\ Ciało\ zostało\ pominięte.}
  Program wykrył ujemną masę (o wartości \emph{-2.1}) w linii 31 danych
  wejściowych. Program ignoruje nieprawidłowe ciało.
\end{itemize}

\end{document}